%&/Users/mmb/bpep/untracked/style_perso/bpepFormatCOLLE
\documentclass[a4paper, 11pt, onside]{extarticle}
\usepackage{/Users/mmb/bpep/fichiers_utiles/raccourcis_communs}

\usepackage{hyperref}
\setmarginsrb{1.8cm}{1.8 cm}{0cm}{0.1cm}{0cm}{0.6cm}{0cm}{0.7cm}
\setmarginsrb{0.9cm}{0.5cm}{0.9cm}{0.5cm}{0.6cm}{0.6cm}{0.6cm}{0.9cm}
\renewcommand{\thesubsection}{\Alph{subsection}}

\renewcommand{\titreExercice}[1]{
    \begin{center}
    \framebox{\Large \textbf{#1} }
    \end{center}
}

\csname endofdump\endcsname

\usepackage{qrcode}
\newcommand{\qr}[1]{
\begin{tikzpicture}[overlay, remember picture]
  \node[xshift=-0cm,yshift=-0cm] at (current page.north east) {
\qrcode[nolink,height=3cm]{mailto:maxence.miguelbrebion@gmail.com?subject=Demande\%20de\%20correction&body=Pour\%20l'exo\%20#1\%20Merci.} };  
  \node[xshift=-0cm,yshift=-1.9cm] at (current page.north east) {\small Pour demander};
  \node[xshift=-0cm,yshift=-2.3cm] at (current page.north east) {\small le corrigé !};
;  
\end{tikzpicture}
}

\newcommand{\titl}[2]{
\clearpage
\begin{center}
    \Huge  \textbf{Sujet de colle \dsNB -#1}
\end{center}
\qr{\dsNB -#1}

\tcols{0.75}{0.24}{
\textbf{Question de cours} : 
\itshape #2
}{}
\bigskip
}

%%%%%%%%%%%%%%%%%%%%%%%%%%%%%%%%%%%%%%%%%%%%%%%%
%%%%%%%%%%%%%%%%%%%% Bloc à modifier %%%%%%%%%%%%%%%%%
%%%%%%%%%%%%%%%%%%%%%%%%%%%%%%%%%%%%%%%%%%%%%%%%

 \toggletrue{corrige}  % décommenter pour passer en mode corrigé
\newcommand{\dsNB}{22}

%%%%%%%%%%%%%%%%%%%%%%%%%%%%%%%%%%%%%%%%%%%%%%%%
%%%%%%%%%%%%%%%%%%%%%%%%%%%%%%%%%%%%%%%%%%%%%%%%
%%%%%%%%%%%%%%%%%%%%%%%%%%%%%%%%%%%%%%%%%%%%%%%%


\begin{document}

\resetQ
\titl{1}{ %Question de cours :
 Expressions des moment \textbf{et} énergie cinétique d'un solide en rotation autour d'un axe fixe $\Delta$ (démonstration avec la somme).
}

\subimport{/Users/mmb/bpep/exercices/DS/oscillation_piston/}{sujet.tex}


\resetQ
\titl{2}{ %Question de cours :
Démonstration de la formule de la pression cinétique pour le GP : 
$$ 
	p = \frac{1}{3}m n^* v^{*2}
$$
où $n^*=N/V$ et $v^*$ représente la vitesse quadratique moyenne des particules
}

\subimport{/Users/mmb/bpep/exercices/Colle/glissement_plaque/}{sujet.tex}


\resetQ
\titl{3}{ %Question de cours :
Démonstration de la formule de la pression cinétique pour le GP : 
$$ 
	p = \frac{1}{3}m n^* v^{*2}
$$
où $n^*=N/V$ et $v^*$ représente la vitesse quadratique moyenne des particules.

}

\subimport{/Users/mmb/bpep/exercices/Colle/fil_tire/}{sujet.tex}


\resetQ
\titl{4}{ %Question de cours :
Obtention de l'équation du mouvement pour le pendule de torsion constitué d'une tige d'axe $\vec e_z$ et soumise à un couple $\vec \Gamma = - C \theta \vec e_z$ et de moment d'inertie $J$. (frottements négligés).
}

\subimport{/Users/mmb/bpep/exercices/Colle/libre_parcours_moyen/}{sujet.tex}




\end{document}