% Il s'agit du template par defaut pour l'affichage d'une preview
% ce fichier ne doit pas être modifié

\documentclass[a4paper, 11pt, onside]{extarticle}
\usepackage{/Users/mbrebion/prepa/bpep/fichiers_utiles/raccourcis_communs}
\usepackage{fancyvrb}
\usepackage{fvextra}

\setmarginsrb{0.9cm}{0.5cm}{0.9cm}{0.5cm}{0.6cm}{0.6cm}{0.6cm}{0.9cm}

 % IMPORTS automatiques
\usepackage{amsmath,mathtools}

 % fin des IMPORTS automatiques

\begin{document}


\vspace{-1cm}
\vspace{-1cm}
\section*{Readme.txt} 
\begin{Verbatim}[breaklines=true]
auteur(s) : Maxence Miguel-Brebion
contibuteur(s) :
source(s) : concours Polytechnique via E. Antoine

thème de l'exercice :
accelerateur de particule linéaire + étude de stabilité/perturbation
- La première partie est un peu technique mais abordable
- La seconde partie est franchement plus compliquée (beaucoup de notations introduites, passage de relation de recurrences à des équations différentielles en utilisant des DLs)
\end{Verbatim}
\vspace{-5mm}
\section*{Mots clés} 
\begin{Verbatim}[breaklines=true]
champ electrique    mécanique    particule chargée    PCSI    probleme    SUP    
\end{Verbatim}
\resetQ
\subimport{/Users/mbrebion/prepa/bpep/exercices/DS/accelerateur_wideroe/}{sujet.tex}




\end{document}